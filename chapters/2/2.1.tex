\renewcommand{\theequation}{\theenumi}

\begin{enumerate}[label=\arabic*.,ref=\thesubsection.\theenumi]
		\numberwithin{equation}{enumi}
	\item Determine if the following sets G with the operation indicated form a group. If not, point out which of the group axioms fail. 
\begin{enumerate}
	\item  $G$ = set of all integers, $a * b = a-b$. 
	\item  $G$ = set of all integers, $a * b = a + b + ab $
	\item  $G$ = set of nonnegative integers, $a * b = a + b$. 
	\item  $G$ = set of all rational numbers $\ne  -1, a * b = a + b + ab $.
	\item  $G$ = set of all rational numbers with denominator divisible by 5 (written so that numerator and denominator are relatively prime), $a * b = a + b$.
	\item  $G$ a set having more than one element, $a * b = a \forall a, b \in G$.

\end{enumerate}
\solution The properties of a group are 
\begin{enumerate}
	\item $a,b \in G \implies a*b \in G$.
	\item $a,b,c \in G \implies a*(b*c) = (a*b)*c \in G$.
		\label{prop:group-def}
	\item $\exists \, e \in G \ni a*i = i*a = a \, \forall \, a \in G$.
		\label{prop:group-def-i}
	\item $a \in G \implies \exists \, b \in G \ni a*b = b*a = i$.
		\label{prop:group-def-inv}
\end{enumerate}
\begin{enumerate}
	\item From
		\ref{prop:group-def},
		\begin{align}
			a*(b*c) &= a-(b-c) = a-b+c
			\\
			(a*b)*c &= (a-b)-c = a-b-c
			\\
			\implies 
			a*(b*c) 
			\ne			
			(a*b)*c
		\end{align}
		Thus, $G$ is not a group.
	\item 
		\label{prop:2.1.1b}
		\begin{enumerate}
	\item 
		From property
		\ref{prop:group-def},
		\begin{align}
			a*(b*c) &= a*(b+c+bc) 
			\\
			&= a+b+c+bc+a(b+c+bc)
			\\
			&= a+b+c + ab+bc+ca +abc
			\\
			(a*b)*c &= \brak{a+b+ab}+c+c\brak{a+b+ab}
			\\
			&= a+b+c+ab+bc+ca+abc
		\end{align}
		Thus, property 
		\ref{prop:group-def}
		is satisfied.
	\item Since
		\begin{align}
			a*i &= a+i+ai
			\\
			i*a &= a+i+ai
		\end{align}
		property
		\ref{prop:group-def-i} is satisfied.
	\item 
		\begin{align}
			a*i &= a+i+ai
			\\
			i*a &= a+i+ai
		\end{align}
		Thus, for 
		property
		\ref{prop:group-def-i} to be satisfied,
		\begin{align}
			i*a &= a
			\\
			\implies a+i+ai = a
			\\
			\text{or, }
			i\brak{1+a} &= 0
			\\
			\implies i = 0
		\end{align}

	\item   If
		\begin{align}
			a*b &= b*a  = i, 
			\\
			a+b+ab &= 0
			\\
			\implies b = -\frac{a}{1+a}
		\end{align}
		which  is not finite for $a = -1$.  Also, $b \notin G$ for $a = 1$.  Thus, property 
		\ref{prop:group-def-inv} is violated and $G$ is not a group.

\end{enumerate}
	\item In this case, for 
		property
		\ref{prop:group-def-i} to be satisfied,
		\begin{align}
			a*i &= i*a = a, 
			\\
			\implies 
			a+i &=a 
			\\
			\text{or, } i = 0
		\end{align}
From property
		\ref{prop:group-def-i},
		\begin{align}
a+b = 0 \implies b = -a
		\end{align}
		Thus, $G$ is a group.
	\item From problem 
		\ref{prop:2.1.1b}, it is easy to verify that $G$ is a group, since we are now considering rational numbers.
	\item From property 
		\ref{prop:group-def-i},
		\begin{align}
			a*i &= a, i*a = i 
			\\
			\implies a = i
		\end{align}
From property 
		\ref{prop:group-def-inv},
		\begin{align}
			a*b &= b*a = i 
			\\
			\implies i*b &= i, b*i = b = i
		\end{align}
		Thus, $G$ has only a single element $i$ which is a contradiction.  So $G$ is not a group.
\end{enumerate}
\item Let $G$ be the set of all mappings	
		\begin{align}
			T_{a,b} \mid T_{a,b}(r) = ar+b, \quad a \ne 0,b,r \in \mathbb{R}, 
		\end{align}
	show that the set $H = {T_{a,b}\mid a = \pm 1, b \in \mathbb{R}}$ forms a group under the $*$ of $G$.
	\\
	\solution 
\begin{enumerate}
	\item 
		\begin{align}
			\brak{T_{a,b}
*
			T_{a,c}}(r) &= a\brak{ar+c}+b
			\\
			&=a^2r + ac+b = r+ac+b
			\\
			&=T_{1,ac+b} \in G
			\label{eq:2.1.2-bc}
		\end{align}
		Similarly, 
		\begin{align}
			\brak{T_{a,c}
*
			T_{a,b}}(r) 
			&=r+ab+c
		\end{align}
	\item 	If 
		\begin{align}
			\brak{T_{a,b}
*
			T_{a,c}}(r) &= T_{a,b}, 
			\\
			r+ab+c &= ar+b
			\\ 
			\implies 
			a &= 1, c = 0
		\end{align}
		Thus, 
		\begin{align}
			i = T_{1,0}
		\end{align}
	\item 	If 
		\begin{align}
			\brak{T_{a,b}
*
			T_{a,c}}(r) &= 
			\brak{T_{a,c}
*
			T_{a,b}}(r) = T_{1,0},
			\\
		r+ab+c
			&=
		r+ac+b = r
		\\
			\implies  b &= \pm c
		\end{align}
	\item From 
		\label{prob:2.1.2}
			\eqref{eq:2.1.2-bc},
		\begin{align}
			T_{a,b}
*
			\brak{
			T_{a,c}(r) 
*
			T_{a,d}}(r) 
		&= 
			T_{a,b}
*
			T_{1,ad+c}
			\\
			&=a\brak{r+ad+b + c}+b 
			\\
			&= ar + ab+ac+b+d
		\end{align}
		Similalrly, 
		\begin{align}
			\brak{
			T_{a,b}
*
			T_{a,c}}(r) 
*
			T_{a,d}(r) 
		&= 
			T_{1,ac+b}
*
			T_{a,d}
			\\
			&=\brak{ar+ d}+ac+b 
			\\
			&= ar + ab+ac+b+d
		\end{align}
		which satisfies the associativity property.
\end{enumerate}
		Thus, $T^{-1}_{a,b} = T_{a,\pm b}$.
		and $G$ is a group.
	\item Let $H \subset G$, for $G$ in problem 
		\ref{prob:2.1.2} and $H = \cbrak{{T_{a,b} \in G \mid a \text{ is rational}, b \text{ any real}}}$.  Show that $H$ is also a group.
	\item Let $K \subset G$, for $G$ in problem 
		\ref{prob:2.1.2} and $K = \cbrak{T_{1,b} \in G \mid  b \in \mathbb{R}}$.  Show that $K$ is an Abelian group.
		\\
		\solution  From \eqref{eq:2.1.2-bc},
		\begin{align}
T_{1,b}
*
			T_{1,c} &= 
T_{1,b+c}  
\\
			&= T_{1,c+b}  
		\end{align}
		Thus, $K$  is an Abelian group.
	\item In Example 9, prove that $g * f = f * g^{-1}$, and that $G$ is a group, is nonabelian, and is of order 8
Let $S = \cbrak{(x, y) \mid x, y real} and consider $f, g \in A(S)$ defined by $f (x, y) = (-x, y)$ and $g(x, y) = (-y, x); f$ is the reflection about the $y$-axis and $g$ is the rotation through $90\degree$ in a counterclockwise direction about the origin. We then define $G = f^ig^j 
\mid i = 0, 1; j = 0, 1, 2, 3}$, and let *
in $G$ be the product ofelements in A(S). Clearly, f2 = g4 = identity mapping; (f* g)(x, y) = (fg)(x, y) = f(g(x, y» = f( - y, x) = (y, x)
and (g*f)(x,y) = g(f(x,y» = g(-x,y) = (-y, -x).
So g * f =1= f * g. We leave it to the reader to verify that g * f = f * g-l and G is a nonabelian group of order 8. This group is called the dihedral group of order 8. [Try to find a formula for (fig j
) * (fSgt) = fagb that expresses a, b in terms of i, j, s, and t.]

\end{enumerate}
