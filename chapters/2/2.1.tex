\renewcommand{\theequation}{\theenumi}

\begin{enumerate}[label=\arabic*.,ref=\thesubsection.\theenumi]
		\numberwithin{equation}{enumi}
	\item Determine if the following sets G with the operation indicated form a group. If not, point out which of the group axioms fail. 
\begin{enumerate}
	\item  $G$ = set of all integers, $a * b = a-b$. 
	\item  $G$ = set of all integers, $a * b = a + b + ab $
	\item  $G$ = set of nonnegative integers, $a * b = a + b$. 
	\item  $G$ = set of all rational numbers $\ne  -1, a * b = a + b + ab $.
	\item  $G$ = set of all rational numbers with denominator divisible by 5 (written so that numerator and denominator are relatively prime), $a * b = a + b$.
	\item  $G$ a set having more than one element, $a * b = a \forall a, b \in G$.

\end{enumerate}
\solution The properties of a group are 
\begin{enumerate}
	\item $a,b \in G \implies a*b \in G$.
	\item $a,b,c \in G \implies a*(b*c) = (a*b)*c \in G$.
		\label{prop:group-def}
	\item $\exists \, e \in G \ni a*i = i*a = a \, \forall \, a \in G$.
		\label{prop:group-def-i}
	\item $a \in G \implies \exists \, b \in G \ni a*b = b*a = i$.
		\label{prop:group-def-inv}
\end{enumerate}
\begin{enumerate}
	\item From
		\ref{prop:group-def},
		\begin{align}
			a*(b*c) &= a-(b-c) = a-b+c
			\\
			(a*b)*c &= (a-b)-c = a-b-c
			\\
			\implies 
			a*(b*c) 
			\ne			
			(a*b)*c
		\end{align}
		Thus, $G$ is not a group.
	\item 
		\label{prop:2.1.1b}
		\begin{enumerate}
	\item 
		From property
		\ref{prop:group-def},
		\begin{align}
			a*(b*c) &= a*(b+c+bc) 
			\\
			&= a+b+c+bc+a(b+c+bc)
			\\
			&= a+b+c + ab+bc+ca +abc
			\\
			(a*b)*c &= \brak{a+b+ab}+c+c\brak{a+b+ab}
			\\
			&= a+b+c+ab+bc+ca+abc
		\end{align}
		Thus, property 
		\ref{prop:group-def}
		is satisfied.
	\item Since
		\begin{align}
			a*i &= a+i+ai
			\\
			i*a &= a+i+ai
		\end{align}
		property
		\ref{prop:group-def-i} is satisfied.
	\item 
		\begin{align}
			a*i &= a+i+ai
			\\
			i*a &= a+i+ai
		\end{align}
		Thus, for 
		property
		\ref{prop:group-def-i} to be satisfied,
		\begin{align}
			i*a &= a
			\\
			\implies a+i+ai = a
			\\
			\text{or, }
			i\brak{1+a} &= 0
			\\
			\implies i = 0
		\end{align}

	\item   If
		\begin{align}
			a*b &= b*a  = i, 
			\\
			a+b+ab &= 0
			\\
			\implies b = -\frac{a}{1+a}
		\end{align}
		which  is not finite for $a = -1$.  Also, $b \notin G$ for $a = 1$.  Thus, property 
		\ref{prop:group-def-inv} is violated and $G$ is not a group.

\end{enumerate}
	\item In this case, for 
		property
		\ref{prop:group-def-i} to be satisfied,
		\begin{align}
			a*i &= i*a = a, 
			\\
			\implies 
			a+i &=a 
			\\
			\text{or, } i = 0
		\end{align}
From property
		\ref{prop:group-def-i},
		\begin{align}
a+b = 0 \implies b = -a
		\end{align}
		Thus, $G$ is a group.
	\item From problem 
		\ref{prop:2.1.1b}, it is easy to verify that $G$ is a group, since we are now considering rational numbers.
	\item 

\end{enumerate}
\end{enumerate}
