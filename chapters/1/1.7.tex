\renewcommand{\theequation}{\theenumi}

\begin{enumerate}[label=\arabic*.,ref=\thesubsection.\theenumi]
		\numberwithin{equation}{enumi}
	\item Multiply
		\begin{enumerate}
			\item $\brak{6-7\j}\brak{8+\j}$
			\item $\brak{\frac{2}{3}+\frac{3}{2}\j}\brak{\frac{2}{3}-\frac{3}{2}\j}$
			\item $\brak{6+7\j}\brak{8-\j}$
		\end{enumerate}
		\solution 
		\begin{enumerate}
			\item 
				\begin{align}
					\brak{6-7\j}\brak{8+\j} &= \myvec{6 & 7 \\ -7 & 6}\myvec{8 \\ 1}
					\\
					&= \myvec{53 \\ -50} = 53 - 50 \j
				\end{align}
			\item 
				\begin{align}
					\brak{\frac{2}{3}+\frac{3}{2}\j}\brak{\frac{2}{3}-\frac{3}{2}\j} &= \myvec{\frac{2}{3} & -\frac{3}{2}\\ \frac{3}{2} &\frac{2}{3}}\myvec{\frac{2}{3} \\ -\frac{3}{2}}
					\\
					&=\myvec{\frac{97}{36} \\ 0} = \frac{97}{36}
				\end{align}
			\item
				\begin{align}
					\brak{6+7\j}{8-\j} &= \sbrak{\brak{6-7\j}{8+\j}}^{*}
					\\
					&= \brak{53 - 50 \j}^{*} = 53 + 50 \j
				\end{align}
		\end{enumerate}
	\item Find $z^{-1}$ for
		\begin{enumerate}
			\item $z = 6+8\j$
			\item $z = 6-8\j$
			\item $z = \frac{1}{\sqrt{2}}\brak{1+\j}$
		\end{enumerate}
		\solution
		\begin{enumerate}
			\item 

				\begin{align}
					z^{-1} = \frac{z^{*}}{\abs{z}^2} = \frac{6-8\j}{100}
				\end{align}
			\item 
				\begin{align}
					z^{-1} =  \frac{6+8\j}{100}
				\end{align}
			\item 
				\begin{align}
					z^{-1} =  \frac{1-\j}{\sqrt{2}}
				\end{align}
		\end{enumerate}
	\item Show that 
				\begin{align}
					\label{eq:1.7.3}
		\brak{z^{*}}^{-1} = \brak{z^{-1}}^{*}
				\end{align}
		\solution Since
				\begin{align}
						zz^{-1} &= 1, 
\\
						\brak{zz^{-1}}^{*} &= 1 
						\\
						\implies 
						\brak{z}^{*}\brak{z^{-1}}^{*} &= 1 
				\end{align}
				yielding
					\eqref{eq:1.7.3}.
				\item Find 
				\begin{align}
					\brak{\cos \theta + \j \sin \theta}^{-1}
				\end{align}


				\solution  
				\begin{align}
					\brak{\cos \theta + \j \sin \theta}^{-1} = \cos \theta - \j \sin \theta
				\end{align}
			\item Verify the following
				\begin{enumerate}
					\item	$\brak{z^{*}}^{*} = z$
					\item	$\brak{z+w}^{*} = z^{*}+w^{*}$
					\item $z + z^{*} = 2 \re{z}$
					\item $z - z^{*} = 2 \j\im{z}$
				\end{enumerate}
				\solution
				\begin{enumerate}
					\item For 
				\begin{align}
					z &= a + \j b, 	
					\\
					z^{*} &= a - \j b, 	
					\\
					\implies \brak{z^{*}}^{*} &= a +\j b = z
				\end{align}
			\item For 
				\begin{align}
					\begin{split}
					z &= z_1 + \j z_2
					\\
					w &= w_1 + \j w_2,
					\end{split}
					\\
					\brak{z+w}^{*} &= \brak{z_1 + \j z_2+ w_1 + \j w_2}^{*}
					\\
					&= \brak{z_1 - \j z_2}+ \brak{w_1 - \j w_2}
					\\
					&= z^{*} + w^{*}
				\end{align}
			\item 	For 
				\begin{align}
					z &= a + \j b, 	
					\\
					z^{*} &= a - \j b, 	
					\\
					\implies \brak{z+z^{*}} &= a +\j b + a - jb
					\\
					&=2a = 2 \re{z}
				\end{align}
			\item 	For 
				\begin{align}
					z &= a + \j b, 	
					\\
					z^{*} &= a - \j b, 	
					\\
					\implies \brak{z-z^{*}} &= a +\j b - a + jb
					\\
					&=2\j b = 2 \im{z}
				\end{align}

				\end{enumerate}
			\item Show that $z$ is real if and only if $z^{*} = z$ and is purely imaginary if and only if $z^{*} = -z$.
				\solution Let
				\begin{align}
					z = a + \j b.
				\end{align}
				Then 
				\begin{align}
					z^{*} = a - \j b.
				\end{align}
				If 
				\begin{align}
					z^{*} &= z,
					\\
					a + \j b &= 
					a - \j b 
					\\
					\implies b = 0
				\end{align}
				and $z$ is real.  If $z$ is real, 
				\begin{align}
					z &= a
					\\
					\implies z^{*} &= a
					\\
					\text{or, } z = z^{*}
				\end{align}
Similarly, the other property can be proved.
\item Verify the commutative law of multiplication $zw  = wz$ in $\mathbb{C}$.
	\\
	\solution
	Let 
				\begin{align}
					z &= a + \j b
					\\
					w &= x - \j y
				\end{align}
				Then
				\begin{align}
					zw &= \myvec{
						a & -b
						\\
						b & a
					}
					\myvec{x \\ y}
					\\
					&=  \myvec{
						x & -y
						\\
						y & x
					}
					\myvec{a \\ b}
					\\
					&= wz
				\end{align}
			\item Show that for $z \ne 0$, $\abs{z}^{-1} = \frac{1}{\abs{z}}$.
				\\
				\solution 
				Let
				\begin{align}
					z = re^{\j \theta}.
				\end{align}
				Then 
				\begin{align}
					z^{-1} &= \frac{1}{r}e^{-\j \theta}
					\\
					\implies \abs{z^{-1}} &= \frac{1}{r}
				\end{align}
			\item Find
\begin{enumerate}
	\item $\abs{6 -4\j}$.
	\item $\abs{\frac{1}{2} +\frac{2}{3}\j}$.
	\item $\abs{\frac{1}{\sqrt{2}}\brak{1+\j}}$
\end{enumerate}
\solution 
\begin{enumerate}
	\item 
				\begin{align}
					\abs{6 -4\j} = \sqrt{6^2 + 4^2} = 2 \sqrt{13}
				\end{align}
			\item  				\begin{align}
					\abs{\frac{1}{2} +\frac{2}{3}\j} = \frac{5}{6}
				\end{align}
\item  				\begin{align}
		\abs{\frac{1}{\sqrt{2}}\brak{1+\j}} = \frac{1}{\sqrt{2}}\abs{\brak{1+\j}} = 1
\end{align}


\end{enumerate}
\item Show that $\abs{z^{*}} = \abs{z}$.
\\
\solution Let 
\begin{align}
	z = re^{\j\theta}
\end{align}
Then 
\begin{align}
	z^{*} &= re^{-\j\theta}
	\\
	\implies \abs{z^{*}} &= r = \abs{z}.
\end{align}
\item Find the polar form for 
\begin{enumerate}
	\item $z = \frac{\sqrt{2}}{2}-\frac{1}{\sqrt{2}}\j$.
	\item $z = 4\j$.
	\item $z = \frac{6}{\sqrt{2}}+ \frac{6}{\sqrt{2}}\j$.
	\item $z = -\frac{13}{2}+\frac{39}{2\sqrt{3}}\j$.
\end{enumerate}
\solution 
\begin{enumerate}
	\item 
		\begin{align}
			\abs{z} &= \sqrt{\brak{\frac{\sqrt{2}}{2}}}^2 + \brak{\frac{1}{\sqrt{2}}}^2.
			\\
			&= 1
		\end{align}
		and 
		\begin{align}
			\angle{z} &= - \tan^{-1} \frac{\frac{1}{\sqrt{2}}}{\frac{1}{\sqrt{2}}}
			\\
			&= \frac{\pi}{4}
		\end{align}
	\item 
		\begin{align}
			\abs{z} = 4, \angle{z} = \frac{\pi}{2}.
		\end{align}
	\item 
		\begin{align}
			\abs{z} = \frac{6}{\sqrt{2}}, \angle{z} =  \frac{\pi}{4}.
		\end{align}
	\item 
		\begin{align}
			\abs{z} &= \frac{13}{2}\sqrt{1 + 3}
			\\
			&= 13
		\end{align}
		and 
		\begin{align}
			\angle{z} &= \pi - \tan^{-1}\frac{\frac{39}{2\sqrt{3}}}{\frac{13}{2}}
			\\
			&= \pi - \tan^{-1}\sqrt{3} = \frac{2\pi}{3}
		\end{align}
\end{enumerate}
\item  Prove that
		\begin{align}
			\brak{\cos\brak{\frac{\theta}{2}}+\j\sin\brak{\frac{\theta}{2}}}^2 = \cos\brak{\theta}+\j\sin\brak{\theta}
		\end{align}
		\solution The L.H.S can be expressed as 
		\begin{align}
			\brak{e^{\j\theta}}^2 = e^{\j\theta}
		\end{align}
	\item Show that
		\begin{align}
			\brak{\frac{1}{2}+\frac{\sqrt{3}}{2}\j}^3 = -1
		\end{align}
		\solution
		\begin{align}
\frac{1}{2}+\frac{\sqrt{3}}{2}\j &= e^{\frac{\j\pi}{3}}
			\\
			\implies \brak{e^{\frac{\j\pi}{3}}}^3 &= e^{\j\pi} = -1
		\end{align}
	\item Show that 
		\begin{align}
			\brak{\cos\brak{\theta}+\j\sin\brak{\theta}}^m = 
			\cos\brak{m\theta}+\j\sin\brak{m\theta}  
			\label{eq:demoivre}
		\end{align}
		for all integers $m$.
		\solution It is easy to verify that 
		\begin{align}
			\brak{ \cos\brak{\theta}+\j\sin\brak{\theta}}^2 = 
			 \cos\brak{2\theta}+\j\sin\brak{2\theta}  
		\end{align}
		Then 
		\begin{multline}
			\brak{\cos\brak{\theta}+\j\sin\brak{\theta}}^{k+1} = 
			\brak{\cos\brak{\theta}+\j\sin\brak{\theta}}^{k}
			\\
			\brak{ 
			\cos\brak{m\theta}+\j\sin\brak{m\theta}  }
			\\
			=
			\cos\sbrak{\brak{k+1}\theta+\j\sin\brak{\brak{k+1}\theta} }
		\end{multline}
		By induction, 
			\eqref{eq:demoivre}
			is proved.
	\item Show that 
		\begin{align}
			\brak{\cos\brak{\theta}+\j\sin\brak{\theta}}^r = 
			\cos\brak{r\theta}+\j\sin\brak{r\theta}  
			\label{eq:demoivre-rational}
		\end{align}
		for all rational numbers $r$.
			\solution 	
			Let 
		\begin{align}
			r = \frac{m}{n},  \brak{\cos\brak{\theta}+\j\sin\brak{\theta}}^{\frac{1}{n}} = \cos\brak{\alpha}+\j\sin\brak{\alpha}
		\end{align}
		Then 
		\begin{align}
			\brak{	\cos\brak{\alpha}+\j\sin\brak{\alpha} }^{n} &= \brak{\cos\brak{\theta}+\j\sin\brak{\theta}}
			\\
			\implies \cos\brak{n\alpha}+\j\sin\brak{n\alpha}  &= \brak{\cos\brak{\theta}+\j\sin\brak{\theta}}
			\\
			\text{or, } \alpha = \frac{\theta}{n}
		\end{align}
		yielding 
		\begin{align}
			\brak{\cos\brak{\theta}+\j\sin\brak{\theta}}^{\frac{1}{n}} = \cos\brak{\frac{\theta}{n}}+\j\sin\brak{\frac{\theta}{n}}
			\label{eq:demoivre-inv}
		\end{align}
			Using \eqref{eq:demoivre} and  
			\eqref{eq:demoivre-inv},
		\begin{align}
\brak{\cos\brak{\theta}+\j\sin\brak{\theta}}^{\frac{m}{n}}
= \cos\brak{\frac{m\theta}{n}}+\j\sin\brak{\frac{m\theta}{n}}
		\end{align}
	\item If $z \in \mathbb{C}$ and $n \ge 1$ is any positive integer, show that there are $n$ distinct complex numbers such that $z = w^n$.
		\solution Let 
		\begin{align}
			z = \cos\brak{\theta}+\j\sin\brak{\theta}
		\end{align}
		then using 
			\eqref{eq:demoivre-inv},
		\begin{align}
			w = \cos\brak{\frac{2\pi k +\theta}{n}}+\j\sin\brak{\frac{2\pi k +\theta}{n}}, k = 0,\dots,n-1
		\end{align}
		which are the distinct roots.
	\item Find the necessary and sufficient condition on $k$ such that 
		\begin{align}
			\brak{\cos\brak{\frac{2\pi k }{n}}+\j\sin\brak{\frac{2\pi k }{n}}}^n = 1 \quad \text{and} 
			\\
			\brak{\cos\brak{\frac{2\pi k }{n}}+\j\sin\brak{\frac{2\pi k }{n}}}^m \ne 1 \quad 0 < m < n
		\end{align}
		\solution From the above equations, using 
			\eqref{eq:demoivre}, 
		\begin{align}
			\frac{mk}{n} \notin \mathbb{Z}
		\end{align}
	\item Viewing the $x-y$ plane as the set of all complex numbers $x+\j y$, show that multiplication by $j$ induces as $90\degree$ rotation of the $x-y$ plan in counterclockwise direction.
		\label{prob:rot}
		\\
		\solution The given multiplication can be expressed using matrices as 
		\begin{align}
			\myvec{
				\cos 90\degree & -\sin 90\degree
				\\
				\sin 90\degree & \cos 90\degree
			}	
			\myvec{x \\ y}
		\end{align}
		which is the multiplication of $\myvec{x\\y}$ with a $90\degree$ rotation matrix.
	\item In problem 
		\eqref{prob:rot}, interpret geometrically what multiplication by the complex number $a + \j b$ does to the $x-y$ plane.
		\\
		\solution The multiplication can be represented as 
		\begin{align}
			\sqrt{a^2+b^2}
			\myvec{
				\cos \theta & -\sin \theta
				\\
				\sin \theta & \cos \theta
			}	
			\myvec{x \\ y}
		\end{align}
		where 
		\begin{align}
			\cos \theta  &= \frac{a}{\sqrt{a^2+b^2}}
			\sin\theta  &= \frac{b}{\sqrt{a^2+b^2}}
		\end{align}
		Geometrically, multiplication by $a+\j b$ results in rotation by $\theta$ and scaling by $\sqrt{a^2+b^2}$.
	\item Prove that 
		\begin{align}
			\abs{z+w}^2+\abs{z-w}^2 = 2\brak{\abs{z}^2 + \abs{w}^2}
			\label{eq:pgm}
		\end{align}
		\solution 
		Since 
		\begin{align}
			\abs{z+w}^2 &= \brak{z+w}^{*}\brak{z+w}
			\\
			&= \abs{z}^2 + \abs{w}^2 +2z^{*}w
		\end{align}
		and 
		\begin{align}
			\abs{z-w}^2 &= \brak{z+w}^{*}\brak{z-w}
			\\
			&= \abs{z}^2 + \abs{w}^2 -2z^{*}w, 
		\end{align}
		\begin{align}
			\abs{z+w}^2+\abs{z-w}^2 
			&= 2\brak{\abs{z}^2 + \abs{w}^2}
		\end{align}
	\item Consider the set $A = a  + b\j, a, b \in \mathbb{Z}$. Prove that there is  1-1 correspondence of $A$ onto $\mathbb{N}$.
	\item If $a$ is a (complex) root of the polynomial
		\begin{align}
			x^{n} +\alpha_1x^{n-1} + \dots + \alpha_{n-1}x + \alpha_n,
		\end{align}
		where the $\alpha_i$ are real, show that $\bar{a}$ must also be a root.
		\\
		\solution From the given information, 
		\begin{align}
			\bar{a}^{n} +\alpha_1\bar{a}^{n-1} + \dots + \alpha_{n-1}\bar{a} + \alpha_n = 0
		\end{align}
		Thus, $\bar{a}$ is also a root of the given polynomial.
	\item Find the necessary and sufficient conditions on $z$ and $w$ in order that
		\begin{align}
			\abs{z+w} = \abs{z} + \abs{w}
		\end{align}
		\solution 
		\begin{align}
			\abs{z+w}^2 
			&= \abs{z}^2 + \abs{w}^2 +2z^{*}w
			\\
			\brak{\abs{z} + \abs{w}}^2 &= \abs{z}^2 + \abs{w}^2 +2\abs{z}\abs{w}
		\end{align}
		If the above expressions are equal, 
		\begin{align}
z^{*}w = \abs{z}\abs{w}
		\end{align}
		which is the desired condition.
	\item Find the necessary and sufficient conditions on $z_i$  in order that
		\begin{align}
			\abs{\sum_{i=1}^{k}z_i} = \sum_{i=1}^{k}\abs{z_i}
		\end{align}
		\solution 
		\begin{align}
			\label{eq:csineq1}
			\abs{\sum_{i=1}^{k}z_i}^2 &= \sum_{i=1}^{k}\abs{z_i}^2 + 2\sum_{i=1,j=1 \atop i \ne j }^{k}{z_i}^* z_j
			\\
			\brak{\sum_{i=1}^{k}\abs{z_i}}^2 &= \sum_{i=1}^{k}\abs{z_i}^2 + 2\sum_{i=1,j=1 \atop i \ne j }^{k}\abs{{z_i}}\abs{z_j}
			\label{eq:csineq2}
		\end{align}
			From \eqref{eq:csineq1} and 
			\eqref{eq:csineq2},
\begin{multline}
			\sum_{i=1}^{k}\abs{z_i}^2 + 2\sum_{i=1,j=1 \atop i \ne j }^{k}{z_i}^* z_j
			\\
			=
\sum_{i=1}^{k}\abs{z_i}^2 + 2\sum_{i=1,j=1 \atop i \ne j }^{k}\abs{{z_i}}\abs{z_j}
\\
\implies 
\sum_{i=1,j=1 \atop i \ne j }^{k}{z_i}^* z_j
\\
= \sum_{i=1,j=1 \atop i \ne j }^{k}\abs{{z_i}}\abs{z_j}
\end{multline}
which is the desired condition.
\item The complex number $\theta$ is said to have {\em order} $n \ge 1$ if $\theta^{n} = 1$ and $\theta^{m} \ne 1$ for $0 < m < n$.  Show that if $\theta $ has order $n$ and $\theta^k = 1$, where $k > 0$, then $n | k$. 
	\\
	\solution From the given information, 
	\begin{align}
		\theta^{n} = \theta^{k} = 1, k \ge n
	\end{align}
	If $n \nmid k, k = mn+p, 0 < p < n$,  Then, 
	\begin{align}
		\theta^{k} = \theta^{mn+p} = \theta^{p} \ne 1,
	\end{align}
	which is a contradiction, Hence, $n \mid k$.
\item Find all complex numbers $\theta$ having order $n$.
	\\
	\solution If 
	\begin{align}
		\theta^{n} = 1,
		\\
		\theta^{n}= e^{\j 2\pi r}, 0 \le r < n 
	\end{align}
	yielding
	\begin{align}
		\theta = \exp\brak{\j \frac{2\pi r}{n}} 0 \le r < n
	\end{align}
\end{enumerate}
