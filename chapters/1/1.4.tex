\renewcommand{\theequation}{\theenumi}

\begin{enumerate}[label=\arabic*.,ref=\thesubsection.\theenumi]
		\numberwithin{equation}{enumi}

	\item If $s1 \ne s_2$ are in $S$, show that there is an $f \in A(S)$ such that $f(s_1) = s_2$.
		\\
		\solution By definition of a 1-1 mapping, it is obvious.
	\item If $s_1 \in S$, let $H = \cbrak{f \in A(S) \mid f(s_1) = s_1}$. Show that: 
		\label{prob:1.4.2}
		\begin{enumerate}
		\item  $i \in H$. 
		\item  If $f, g \in H$, then $fg \in H$. 
		\item  If $f \in H$, then $f^{-1} \in H$.
		\end{enumerate}
		\solution 
		\begin{enumerate}
			\item $\because i(s_1) = s_1, i \in H$.
			\item $fg(s_1)= f(s_1) = s_1.$
			\item $f(s_1) = s_1 \implies f^{-1}(s_1) = s_1 \implies f^{-1} \in H$.
		\end{enumerate}
	\item Suppose that $s_1 \ne s_2$ are in $S$ and $f(s_1) = s_2$, where $f \in A (S)$. Then if $H$ is as in Problem
		\ref{prob:1.4.2}
		 and $K = \cbrak{g \in A (S) \mid g(s_2) = s_2}$, show that:
\begin{enumerate}
	\item  If $g \in K$, then $f^{-1}gf \in H$. 
	\item  If $h \in H$, then there is some $g \in K$ such that $h = f^{-1}gf$.
\end{enumerate}
\solution
\begin{enumerate}
	\item  From the given information,
		\begin{align}
			f^{-1}gf(s_1) =  
			f^{-1}g(s_2) =   
			f^{-1}(s_2) =  s_1 
		\end{align}
		Hence, 
		\begin{align}
			f^{-1}gf  \in H 
		\end{align}
	\item  The $h$ was found in the previous part.
\end{enumerate}
\item If $f, g, h \in A(S)$, show that $(f^{-1}gf)(f^{-1}hf) = f^{-1}(gh)f$. What can you say about $(f^{-1}gf)^n$?
\solution  From the given information, 
		\begin{align}
			(f^{-1}gf)(f^{-1}hf) &= 
			f^{-1}g\brak{ff^{-1}}hf 
\\
			&			= 
			f^{-1}\brak{gh}f  
		\end{align}
		Similarly,
		\begin{align}
			\brak{f^{-1}gf}^n = 
			f^{-1}g^n  f
		\end{align}
	\item If $f, g \in A (S)$ and $fg = gf$, show that:
		\label{prob:1.4.5}
\begin{enumerate}
	\item  $(fg)^2 = f^2g^2$.
	\item  $(fg)^{-1} = f^{-1}g^{-1}$.
\end{enumerate}
\solution From the given information, 
\begin{enumerate}
	\item  
		\begin{align}
			(fg)^2 &= \brak{fg}\brak{fg}
			\\
			 &= \brak{fg}\brak{gf} = fg^2f
			\\
			 &=  f\brak{fg^2} = f^2g^2
		\end{align}
	\item  Since 
		\begin{align}
			\brak{fg}^{-1}f g&= i,
			\\
			\brak{fg}^{-1}fgg^{-1} &= g^{-1}
			\\
	\implies 	
			\brak{fg}^{-1}f &= g^{-1}
			\\
			\implies \brak{fg}^{-1} ff^{-1}&= g^{-1}f^{-1}
			\\
			\text{or, }\brak{fg}^{-1} &= g^{-1}f^{-1}
		\end{align}

\end{enumerate}
\item Push the result of Problem 
		\ref{prob:1.4.5}
, for the same $f$ and $g$, to show that 
		\begin{align}
			\label{eq:1.4.5}
			\brak{fg}^m = f^mg^m 
		\end{align}
for all integers $m$.
\\
		\solution Using induction, 
		\begin{align}
			\brak{fg}^{m+1} &= 
			\brak{fg}^{m}(fg) 
			\\
			&= f^mg^m fg = 
f^mg^m gf 
			\\
			&= 
ff^mg^m g 
		\end{align}
		yielding
			\eqref{eq:1.4.5}.
		\item
		\item If $f, g \in A (S)$ and $(fg)^2 = f^2g^2$, prove that $fg = gf$.
			\\
			\solution 
		\begin{align}
			(fg)^2 &= f^2g^2
			\\
			\implies fgfg &= ffgg
			\\
			\implies f^{-1}fgfg &= f^{-1}ffgg
			\\
			\implies gfg &= fgg
			\\
			\implies gf gg^{-1}&= fggg^{-1}
		\end{align}
		yielding the desired result.

\end{enumerate}
