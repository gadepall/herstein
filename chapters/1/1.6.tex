\renewcommand{\theequation}{\theenumi}

\begin{enumerate}[label=\arabic*.,ref=\thesubsection.\theenumi]
		\numberwithin{equation}{enumi}
	\item Prove that 
		\begin{align}
			1^2 + 2^2 + 3^2 + \cdots + n^2 = \frac{n\brak{n+1}\brak{2n+1}}{6}
		\end{align}
		by induction.
		\\
		\solution $P(n+1)$ is 
		\begin{multline}
			1^2 + 2^2 + 3^2 + \cdots + n^2 +\brak{n+1}^2 
			\\
			= \frac{n\brak{n+1}\brak{2n+1}}{6}+\brak{n+1}^2
			\\
			= \brak{n+1}\brak{\frac{2n^2+7n + 7}{6}}
			\\
			= \frac{\brak{n+1}\brak{n+2}\brak{2n+3}}{6}
		\end{multline}
		which is true.  Hence, the given proposition is true for all $n \ge 1$
	\item Prove that 
		\begin{align}
			1^3 + 2^3 + 3^3 + \cdots + n^3 = \sbrak{\frac{n\brak{n+1}}{2}}^2
		\end{align}
		by induction.
		\\
		\solution $P(n+1)$ is 
		\begin{multline}
			1^3 + 2^3 + 3^3 + \cdots + n^3 +\brak{n+1}^3 
			\\
			= \sbrak{\frac{n\brak{n+1}}{2}}^2+\brak{n+1}^3
			\\
			= \brak{n+1}^2\brak{\frac{n^2+4n + 4}{4}}
			\\
			= \sbrak{\frac{\brak{n+1}\brak{n+2}}{2}}^2
		\end{multline}
		which is true.  Hence, the given proposition is true for all $n \ge 1$.
	\item Prove that a set having $n \ge 2$ elements has $\frac{n\brak{n-1}}{2}$ subsets having exactly 2 elements.
	\item Prove that a set having $n \ge 3$ elements has $\frac{n\brak{n-1}\brak{n-2}}{3}$ subsets having exactly 3 elements.
	\item If $n \ge 4$ and $S$ is a set having $n$ elements, guess how many subsets having exactly 4 elements are there in $S$.  Then verify your guess using mathematical induction.
	\item If $p$ is a prime and $p \mid \brak{a_1a_2a_3 \dots a_n}$, then prove using induction that $p \mid a_i$ for some $i$ with $1 \le i \le n$.  
	\item If $a \ne 1$, prove that 
		\begin{align}
			1 + a + a^2 + \cdots + a^{n} = \frac{\brak{a^{n+1}-1}}{a-1}
		\end{align}
		by induction.
		\\
		\solution $P\brak{n+1}$ can be expressed as 
		\begin{multline}
			1 + a + a^2 + \cdots + a^{n} +a^{n+1}
			\\
			= \frac{\brak{a^{n+1}-1}}{a-1} + a^{n+1}
			\\
			=\frac{\brak{a^{n+2}-1}}{a-1} 
		\end{multline}
		upon simplification. Hence, the given proposition is true for all $n \ge 1$.
	\item By induction, show that 
		\begin{multline}
			\frac{1}{1\cdot 2}+
			\frac{1}{2\cdot 3}+
\cdots +
			\frac{1}{n\cdot \brak{n+1}}
			\\
			= \frac{n}{n+1}
		\end{multline}
		\solution $P(n+1)$ can be expressed as
		\begin{multline}
			\frac{1}{1\cdot 2}+
			\frac{1}{2\cdot 3}+
\cdots +
			\frac{1}{n\cdot \brak{n+1}}+
			\frac{1}{\brak{n+1}\cdot \brak{n+2}}
			\\
			= \frac{n}{n+1}+ \frac{1}{\brak{n+1}\cdot \brak{n+2}}
			\\
			=
			\frac{1}{n+1}\sbrak{n+ \frac{1}{n+2}}
			\\
			= \frac{n+1}{n+2}
		\end{multline}
		upon simplification. Hence, the given proposition is true for all $n \ge 1$.
	\item Suppose that $P(n)$ is a proposition about positive integers $n$ such that $P\brak{n_0}$ is valid, and if $P(k)$ is true, so must be $P(k+1)$. What can you say about $P(n)$?  Prove your statement.
	\item Let $P(n)$ be a proposition about integers $n$ such that $P(1)$ is true and such that if $P(j)$ is true for all positive integers $j < k$, then $P(k)$ is true.  Prove that $P(n)$ is true for all positive integers $n$.
	\item Given an example of  a proposition that is {\em not} true for any positive integer, yet for which the induction step holds.
	\item Prove by induction that a set having $n$ elements has exactly $2^n$ subsets.
	\item Prove by induction on $n$ that $n^3-n$ is always divisible by 3.
		\\
		\solution $P(n+1)$ can be expressed as 
		\begin{align}
			\brak{n+1}^3 -\brak{n+1} &= n^3 + 3n^2 + 3n +1 -n -1
			\\
			&= n^3-n +3\brak{n^2+n}
		\end{align}
		which is divisible by 3. Hence, the given proposition is true for all $n \ge 1$.
	\item If $p$ is a prime number, then prove that $n^p -n$ is always divisible by p.
\\
\solution $P(n+1)$ can be expressed as 
		\begin{align}
			\brak{n+1}^p -\brak{n+1} &= n^p  -n  + p \sum_{k=1}^{n}\comb{n}{k}p^{k-1}
			\\
			&= n^3-n +3\brak{n^2+n}
		\end{align}

\end{enumerate}
