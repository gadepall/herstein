\renewcommand{\theequation}{\theenumi}

\begin{enumerate}[label=\arabic*.,ref=\thesubsection.\theenumi]
		\numberwithin{equation}{enumi}

	\item For the given sets $S, T$ determine if a mapping $f:S\to T$ is clearly and unambiguously defined; if not, say why not.
		\label{prob:1.3.1}
\begin{enumerate}
	\item $S =$ set of all women, $T = $ set of all men, $f(s) = $ husband of s.
	\item $S = $ set of all positive integers, $T = S, f(s) = s-1$.
	\item $S$ = set of positive integers, $T$ = set of nonnegative integers, $f(s) = s -1$.
	\item $S$ = set of nonnegative integers, $T = S, f(s) = s - 1$.
	\item $S$ = set of all integers, $T = S, f(s) = s - 1$.
	\item  $S$ =set of all real numbers, $T = S, f(s) = \sqrt{s}$.
	\item	$S$ = set of all positive real numbers, $T = S, f(s) = \sqrt{s}$.	
\end{enumerate}
\solution
\begin{enumerate}
	\item Not all women have husbands.  So the mapping is not clearly defined.
	\item For every integer $s, s-1$ is an integer.  So the mapping is defined.
	\item $0 \notin S$, so the mapping is defined.
	\item $f(0) = -1 \notin S$.  So the mapping is not defined.
	\item $f(-1) \notin S$, so the mapping is not defined.
	\item $f(s) \in S \forall S$.  So the mapping is defined.
\end{enumerate}
\item 
	In those parts of Problem  
		\ref{prob:1.3.1}
where $f$ does define a function, determine if it is 1-1, onto, or both.
\solution
\begin{enumerate}
	\item For $f(s) = s-1, s \in \mathbb{Z}$, the mapping is a bijection.
	\item For $s \in \mathbb{N}, f(s) = s-1 \in  \mathbb{W}$, the mapping is a bijection.
	\item For $s \in S, f(s) \in S$ and vice-versa.  So the mapping is a bijection.
\end{enumerate}
\item If $f$ is a 1-1 mapping of $S$ onto $T$, prove that $f^{- 1}$ is a 1-1 mapping of $T$ onto $S$.
	\\
		\solution  By definition,  
		\begin{align}
			\label{eq:1.3.one-one}
			\begin{split}
	s_1 = s_2 \in S	\implies	f(s_1) = f(s_2) \in T
	\\
			t_1 = t_2 \in T \implies \exists s_1 = s_2 \in S \ni f(s_1) = f(s_2).
			\end{split}
		\end{align}
Let $g = f^{-1}$. Then, 
		\begin{align}
			f(s_i) = t_i \implies g(t_i) = s_i.
		\end{align}
		From 
			\eqref{eq:1.3.one-one}, 
		\begin{align}
			\label{eq:1.3.one-one-inv}
			\begin{split}
				g(t_1) = g(t_2) \in S	\implies	 t_1= t_2 \in T
	\\
			t_1 = t_2 \in T \implies \exists g(t_1) = g(t_2) \in S 
			\end{split}
		\end{align}
			\eqref{eq:1.3.one-one-inv} shows that $g = f^{-1}$ is also 1-1.
		\item 
			If $f$ is a 1-1 mapping of $S$ onto $T$, prove that $f^{- 1}\circ f = i_S$.
			\\
			\solution For $s \in S, t \in T$,
			\begin{align}
				f(s) = t \implies g(t) = s
				\\
				\text{or, }g\circ f(s) = s \implies \brak{g\circ f} = i_S \qed
			\end{align}
		\item If $g: S \rightarrow T$ and $f: T\rightarrow  U$ are both onto, then $f\circ g: S \rightarrow   U$ is
also onto.
\\

\solution  From the given information, 
\begin{align}
	g(S) = T, f(T) = U
	\\
	\implies \brak{f\circ g} (S) = U \qed
\end{align}
\item  If $f: S   \rightarrow T$ is onto and $g : T \rightarrow U$ and $h : T  \rightarrow U$ are such that $g\circ f = h \circ f$,  prove that $g = h$.
	\\
	\solution From the given information, 
	\begin{align}
		g\circ f - h \circ f &= 0 
		\\
		\implies \brak{g-h}\circ f &=0
		\\
		\text{or, } g = h
	\end{align}
\item If $g: S  \rightarrow T, h : S  \rightarrow T$, and if $f: T  \rightarrow U$ is 1-1, show that if $f  \circ  g = f   \circ  h$, then $g = h$.
\item Let $S$ be the set of all integers and $T = \cbrak{1, -1}; f: S  \rightarrow T$ is defined by 
	\begin{align}
		f(s)=
	\begin{cases}
 1 & s \text{ even}
		\\
		-1 & s \text{ odd } .
	\end{cases}
	\end{align}
\begin{enumerate}
\item Does this define a function from $S$ to $T$? 
\item  Show that 
	\begin{align}
	f(s_1 + s_2) = f(s_1)f(s_2)
	\end{align}
What does this say about the integers?
\item  Is $f(s_1s_2) = f(s_1)f(s_2)$ also true?
\end{enumerate}
\solution
\begin{enumerate}
	\item Yes, $f$ is a function.
	\item  See Table 
			\ref{table:1.3.8}.
		\begin{table}[!h]
			\centering
			\input{tables/1.3.8.tex}
			\caption{}
			\label{table:1.3.8}
		\end{table}
	\item No. If $s_1,s_2$ are odd, 
	\begin{align}
		s_1s_2 & \text{ odd}
		\\
		f(s_1s_2) &= -1 \ne f(s_1)f(s_2)
	\end{align}
\end{enumerate}
\item Let $S$ be the set of all real numbers.  Define 
	\begin{align}
		f&:S \rightarrow S | f(s) = s^2,
		\\
		g&:S \rightarrow S | g(s) = s+1,
	\end{align}
\begin{enumerate}
	\item Find $f \circ g$.
	\item Find $g \circ f$.
	\item Is $f \circ g=g \circ f$?
\end{enumerate}
\solution 
\begin{enumerate}
	\item 
	\begin{align}
		\label{eq:1.3.9-1}
		\brak{f \circ g}(s) = \brak{s+1}^2
	\end{align}
	\item 
	\begin{align}
		\label{eq:1.3.9-2}
		\brak{g \circ f}(s) = s^2+1
	\end{align}
	\item From 
		\eqref{eq:1.3.9-2}
		and
		\eqref{eq:1.3.9-2}
$f \circ g \ne g \circ f$.
\end{enumerate}
\item Let $S$ be the set of all real numbers and for $a, b \in S$, where $a \ne 0$; define $f_{a,b}(s) = as + b$.
\begin{enumerate}
	\item  Show that $f_{a,b} \circ f_{c,d} = f_{u,v}$ for some real $u, v$. Give explicit values for $u, v$ in terms of $a, b, c$ and $d$.
	\item  Is $f_{a,b} \circ f_{c,d} = f_{c,d} \circ f_{a,b}$ always?
	\item  Find all $f_{a,b}$ such that $f_{a,b} \circ f_{1,1} = f_{1,1} \circ f_{a,b}$.
	\item  Show that $f^{-1}$ exists and find its form.
\end{enumerate}
\solution 
\begin{enumerate}
	\item  
		\begin{align}
			\label{eq:1.3.10-1}
			f_{a,b} \circ f_{c,d} &= c\brak{as+b}+d
			\\
			&=cas+cb+d 
			\\
			&= us+v
			\\
			\implies u = ca, v = bc + d
		\end{align}
	\item  From
			\eqref{eq:1.3.10-1},
		\begin{align}
			\label{eq:1.3.10-2}
			f_{c,d}\circ f_{a,b} 
			&=cas+ad+ b
		\end{align}
		Thus, from 
			\eqref{eq:1.3.10-1}
			and 
			\eqref{eq:1.3.10-2}
		\begin{align}
			f_{a,b} \circ f_{c,d} &= f_{c,d} \circ f_{a,b} 
			\\
			\implies bc+d = ad+b
			\label{eq:1.3.10-3}
		\end{align}
	\item  From 
			\eqref{eq:1.3.10-3},
		\begin{align}
			f_{a,b} \circ f_{1,1} &= f_{1,1} \circ f_{a,b}
			\\
			\implies as+b+1 &= as+a+b
			\\
			\text{or, } a = 1.
		\end{align}
		Thus, 
		\begin{align}
			f_{a,b}  = s+b
		\end{align}
	\item  From the definition,
		\begin{align}
			f_{a,b}(s)  &= as+b
			\\
			\implies s &=\frac{f_{a,b}(s) -b}{a}
			\\
			\text{or, }f^{-1}_{a,b}(s) &= \frac{s-b}{a}
		\end{align}
\end{enumerate}
\item Let $S$ be the set of all positive integers. Define $f: S  \rightarrow S$ by $f(l) = 2, f(2) = 3, f(3) = 1$ and $f(s) = s$ for any other $s \in S$. Show that $f o f o f = i_S$. What is $f^{-1}$ in this case?
	\\
	\solution For
$s \in \cbrak{1,2,3}$, it is obvious.  For
	$s \notin \cbrak{1,2,3}$,
		\begin{align}
			\brak{f\circ f} (s) &= f(s) = s
			\\
			\implies \brak{f\circ f\circ f} (s) &=  s \qed
		\end{align}

\end{enumerate}
