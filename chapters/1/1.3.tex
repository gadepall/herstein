\renewcommand{\theequation}{\theenumi}

\begin{enumerate}[label=\arabic*.,ref=\thesubsection.\theenumi]
		\numberwithin{equation}{enumi}

	\item For the given sets $S, T$ determine if a mapping $f:S\to T$ is clearly and unambiguously defined; if not, say why not.
		\label{prob:1.3.1}
\begin{enumerate}
	\item $S =$ set of all women, $T = $ set of all men, $f(s) = $ husband of s.
	\item $S = $ set of all positive integers, $T = S, f(s) = s-1$.
	\item $S$ = set of positive integers, $T$ = set of nonnegative integers, $f(s) = s -1$.
	\item $S$ = set of nonnegative integers, $T = S, f(s) = s - 1$.
	\item $S$ = set of all integers, $T = S, f(s) = s - 1$.
	\item  $S$ =set of all real numbers, $T = S, f(s) = \sqrt{s}$.
	\item	$S$ = set of all positive real numbers, $T = S, f(s) = \sqrt{s}$.	
\end{enumerate}
\solution
\begin{enumerate}
	\item Not all women have husbands.  So the mapping is not clearly defined.
	\item For every integer $s, s-1$ is an integer.  So the mapping is defined.
	\item $0 \notin S$, so the mapping is defined.
	\item $f(0) = -1 \notin S$.  So the mapping is not defined.
	\item $f(-1) \notin S$, so the mapping is not defined.
	\item $f(s) \in S \forall S$.  So the mapping is defined.
\end{enumerate}
\item 
	In those parts of Problem  
		\ref{prob:1.3.1}
where $f$ does define a function, determine if it is 1-1, onto, or both.
\solution
\begin{enumerate}
	\item For $f(s) = s-1, s \in \mathbb{Z}$, the mapping is a bijection.
	\item For $s \in \mathbb{N}, f(s) = s-1 \in  \mathbb{W}$, the mapping is a bijection.
	\item For $s \in S, f(s) \in S$ and vice-versa.  So the mapping is a bijection.
\end{enumerate}
\item If $f$ is a 1-1 mapping of $S$ onto $T$, prove that $f^{- 1}$ is a 1-1 mapping of $T$ onto $S$.
	\\
		\solution  By definition,  
		\begin{align}
			\label{eq:1.3.one-one}
			\begin{split}
	s_1 = s_2 \in S	\implies	f(s_1) = f(s_2) \in T
	\\
			t_1 = t_2 \in T \implies \exists s_1 = s_2 \in S \ni f(s_1) = f(s_2).
			\end{split}
		\end{align}
Let $g = f^{-1}$. Then, 
		\begin{align}
			f(s_i) = t_i \implies g(t_i) = s_i.
		\end{align}
		From 
			\eqref{eq:1.3.one-one}, 
		\begin{align}
			\label{eq:1.3.one-one-inv}
			\begin{split}
				g(t_1) = g(t_2) \in S	\implies	 t_1= t_2 \in T
	\\
			t_1 = t_2 \in T \implies \exists g(t_1) = g(t_2) \in S 
			\end{split}
		\end{align}
			\eqref{eq:1.3.one-one-inv} shows that $g = f^{-1}$ is also 1-1.
		\item 
			If $f$ is a 1-1 mapping of $S$ onto $T$, prove that $f^{- 1}\circ f = i_S$.
			\\
			\solution For $s \in S, t \in T$,
			\begin{align}
				f(s) = t \implies g(t) = s
				\\
				\text{or, }g\circ f(s) = s \implies \brak{g\circ f} = i_S \qed
			\end{align}
		\item If $g: S \rightarrow T$ and $f: T\rightarrow  U$ are both onto, then $f\circ g: S \rightarrow   U$ is
also onto.
\\

\solution  From the given information, 
\begin{align}
	g(S) = T, f(T) = U
	\\
	\implies \brak{f\circ g} (S) = U \qed
\end{align}
\item  If $f: S   \rightarrow T$ is onto and $g : T \rightarrow U$ and $h : T  \rightarrow U$ are such that $g\circ f = h \circ f$,  prove that $g = h$.
	\\
	\solution From the given information, 
	\begin{align}
		g\circ f - h \circ f &= 0 
		\\
		\implies \brak{g-h}\circ f &=0
		\\
		\text{or, } g = h
	\end{align}
\item If $g: S  \rightarrow T, h : S  \rightarrow T$, and if $f: T  \rightarrow U$ is 1-1, show that if $f  \circ  g = f   \circ  h$, then $g = h$.
\item Let $S$ be the set of all integers and $T = \cbrak{1, -1}; f: S  \rightarrow T$ is defined by 
	\begin{align}
		f(s)=
	\begin{cases}
 1 & s \text{ even}
		\\
		-1 & s \text{ odd } .
	\end{cases}
	\end{align}
\begin{enumerate}
\item Does this define a function from $S$ to $T$? 
\item  Show that 
	\begin{align}
	f(s_1 + s_2) = f(s_1)f(s_2)
	\end{align}
What does this say about the integers?
\item  Is $f(s_1s_2) = f(s_1)f(s_2)$ also true?
\end{enumerate}
\solution
\begin{enumerate}
	\item Yes, $f$ is a function.
	\item  See Table 
			\ref{table:1.3.8}.
		\begin{table}[!h]
			\centering
			%%%%%%%%%%%%%%%%%%%%%%%%%%%%%%%%%%%%%%%%%%%%%%%%%%%%%%%%%%%%%%%%%%%%%%
%%                                                                  %%
%%  This is the header of a LaTeX2e file exported from Gnumeric.    %%
%%                                                                  %%
%%  This file can be compiled as it stands or included in another   %%
%%  LaTeX document. The table is based on the longtable package so  %%
%%  the longtable options (headers, footers...) can be set in the   %%
%%  preamble section below (see PRAMBLE).                           %%
%%                                                                  %%
%%  To include the file in another, the following two lines must be %%
%%  in the including file:                                          %%
%%        \def\inputGnumericTable{}                                 %%
%%  at the beginning of the file and:                               %%
%%        \input{name-of-this-file.tex}                             %%
%%  where the table is to be placed. Note also that the including   %%
%%  file must use the following packages for the table to be        %%
%%  rendered correctly:                                             %%
%%    \usepackage[latin1]{inputenc}                                 %%
%%    \usepackage{color}                                            %%
%%    \usepackage{array}                                            %%
%%    \usepackage{longtable}                                        %%
%%    \usepackage{calc}                                             %%
%%    \usepackage{multirow}                                         %%
%%    \usepackage{hhline}                                           %%
%%    \usepackage{ifthen}                                           %%
%%  optionally (for landscape tables embedded in another document): %%
%%    \usepackage{lscape}                                           %%
%%                                                                  %%
%%%%%%%%%%%%%%%%%%%%%%%%%%%%%%%%%%%%%%%%%%%%%%%%%%%%%%%%%%%%%%%%%%%%%%



%%  This section checks if we are begin input into another file or  %%
%%  the file will be compiled alone. First use a macro taken from   %%
%%  the TeXbook ex 7.7 (suggestion of Han-Wen Nienhuys).            %%
\def\ifundefined#1{\expandafter\ifx\csname#1\endcsname\relax}


%%  Check for the \def token for inputed files. If it is not        %%
%%  defined, the file will be processed as a standalone and the     %%
%%  preamble will be used.                                          %%
\ifundefined{inputGnumericTable}

%%  We must be able to close or not the document at the end.        %%
	\def\gnumericTableEnd{\end{document}}


%%%%%%%%%%%%%%%%%%%%%%%%%%%%%%%%%%%%%%%%%%%%%%%%%%%%%%%%%%%%%%%%%%%%%%
%%                                                                  %%
%%  This is the PREAMBLE. Change these values to get the right      %%
%%  paper size and other niceties.                                  %%
%%                                                                  %%
%%%%%%%%%%%%%%%%%%%%%%%%%%%%%%%%%%%%%%%%%%%%%%%%%%%%%%%%%%%%%%%%%%%%%%

	\documentclass[12pt%
			  %,landscape%
                    ]{report}
       \usepackage[latin1]{inputenc}
       \usepackage{fullpage}
       \usepackage{color}
       \usepackage{array}
       \usepackage{longtable}
       \usepackage{calc}
       \usepackage{multirow}
       \usepackage{hhline}
       \usepackage{ifthen}

	\begin{document}


%%  End of the preamble for the standalone. The next section is for %%
%%  documents which are included into other LaTeX2e files.          %%
\else

%%  We are not a stand alone document. For a regular table, we will %%
%%  have no preamble and only define the closing to mean nothing.   %%
    \def\gnumericTableEnd{}

%%  If we want landscape mode in an embedded document, comment out  %%
%%  the line above and uncomment the two below. The table will      %%
%%  begin on a new page and run in landscape mode.                  %%
%       \def\gnumericTableEnd{\end{landscape}}
%       \begin{landscape}


%%  End of the else clause for this file being \input.              %%
\fi

%%%%%%%%%%%%%%%%%%%%%%%%%%%%%%%%%%%%%%%%%%%%%%%%%%%%%%%%%%%%%%%%%%%%%%
%%                                                                  %%
%%  The rest is the gnumeric table, except for the closing          %%
%%  statement. Changes below will alter the table's appearance.     %%
%%                                                                  %%
%%%%%%%%%%%%%%%%%%%%%%%%%%%%%%%%%%%%%%%%%%%%%%%%%%%%%%%%%%%%%%%%%%%%%%

\providecommand{\gnumericmathit}[1]{#1} 
%%  Uncomment the next line if you would like your numbers to be in %%
%%  italics if they are italizised in the gnumeric table.           %%
%\renewcommand{\gnumericmathit}[1]{\mathit{#1}}
\providecommand{\gnumericPB}[1]%
{\let\gnumericTemp=\\#1\let\\=\gnumericTemp\hspace{0pt}}
 \ifundefined{gnumericTableWidthDefined}
        \newlength{\gnumericTableWidth}
        \newlength{\gnumericTableWidthComplete}
        \newlength{\gnumericMultiRowLength}
        \global\def\gnumericTableWidthDefined{}
 \fi
%% The following setting protects this code from babel shorthands.  %%
 \ifthenelse{\isundefined{\languageshorthands}}{}{\languageshorthands{english}}
%%  The default table format retains the relative column widths of  %%
%%  gnumeric. They can easily be changed to c, r or l. In that case %%
%%  you may want to comment out the next line and uncomment the one %%
%%  thereafter                                                      %%
\providecommand\gnumbox{\makebox[0pt]}
%%\providecommand\gnumbox[1][]{\makebox}

%% to adjust positions in multirow situations                       %%
\setlength{\bigstrutjot}{\jot}
\setlength{\extrarowheight}{\doublerulesep}

%%  The \setlongtables command keeps column widths the same across  %%
%%  pages. Simply comment out next line for varying column widths.  %%
\setlongtables

\setlength\gnumericTableWidth{%
	70pt+%
	70pt+%
	70pt+%
	70pt+%
0pt}
\def\gumericNumCols{4}
\setlength\gnumericTableWidthComplete{\gnumericTableWidth+%
         \tabcolsep*\gumericNumCols*2+\arrayrulewidth*\gumericNumCols}
\ifthenelse{\lengthtest{\gnumericTableWidthComplete > \linewidth}}%
         {\def\gnumericScale{1*\ratio{\linewidth-%
                        \tabcolsep*\gumericNumCols*2-%
                        \arrayrulewidth*\gumericNumCols}%
{\gnumericTableWidth}}}%
{\def\gnumericScale{1}}

%%%%%%%%%%%%%%%%%%%%%%%%%%%%%%%%%%%%%%%%%%%%%%%%%%%%%%%%%%%%%%%%%%%%%%
%%                                                                  %%
%% The following are the widths of the various columns. We are      %%
%% defining them here because then they are easier to change.       %%
%% Depending on the cell formats we may use them more than once.    %%
%%                                                                  %%
%%%%%%%%%%%%%%%%%%%%%%%%%%%%%%%%%%%%%%%%%%%%%%%%%%%%%%%%%%%%%%%%%%%%%%

\ifthenelse{\isundefined{\gnumericColA}}{\newlength{\gnumericColA}}{}\settowidth{\gnumericColA}{\begin{tabular}{@{}p{70pt*\gnumericScale}@{}}x\end{tabular}}
\ifthenelse{\isundefined{\gnumericColB}}{\newlength{\gnumericColB}}{}\settowidth{\gnumericColB}{\begin{tabular}{@{}p{70pt*\gnumericScale}@{}}x\end{tabular}}
\ifthenelse{\isundefined{\gnumericColC}}{\newlength{\gnumericColC}}{}\settowidth{\gnumericColC}{\begin{tabular}{@{}p{70pt*\gnumericScale}@{}}x\end{tabular}}
\ifthenelse{\isundefined{\gnumericColD}}{\newlength{\gnumericColD}}{}\settowidth{\gnumericColD}{\begin{tabular}{@{}p{70pt*\gnumericScale}@{}}x\end{tabular}}

\begin{tabular}[c]{%
	b{\gnumericColA}%
	b{\gnumericColB}%
	b{\gnumericColC}%
	b{\gnumericColD}%
	}

%%%%%%%%%%%%%%%%%%%%%%%%%%%%%%%%%%%%%%%%%%%%%%%%%%%%%%%%%%%%%%%%%%%%%%
%%  The tabular options. (Caption, headers... see Goosens, p.124) %%
%	\caption{The Table Caption.}             \\	%
% \hline	% Across the top of the table.
%%  The rest of these options are table rows which are placed on    %%
%%  the first, last or every page. Use \multicolumn if you want.    %%

%%  Header for the first page.                                      %%
%	\multicolumn{4}{c}{The First Header} \\ \hline 
%	\multicolumn{1}{c}{colTag}	%Column 1
%	&\multicolumn{1}{c}{colTag}	%Column 2
%	&\multicolumn{1}{c}{colTag}	%Column 3
%	&\multicolumn{1}{c}{colTag}	\\ \hline %Last column
%	\endfirsthead

%%  The running header definition.                                  %%
%	\hline
%	\multicolumn{4}{l}{\ldots\small\slshape continued} \\ \hline
%	\multicolumn{1}{c}{colTag}	%Column 1
%	&\multicolumn{1}{c}{colTag}	%Column 2
%	&\multicolumn{1}{c}{colTag}	%Column 3
%	&\multicolumn{1}{c}{colTag}	\\ \hline %Last column
%	\endhead

%%  The running footer definition.                                  %%
%	\hline
%	\multicolumn{4}{r}{\small\slshape continued\ldots} \\
%	\endfoot

%%  The ending footer definition.                                   %%
%	\multicolumn{4}{c}{That's all folks} \\ \hline 
%	\endlastfoot
%%%%%%%%%%%%%%%%%%%%%%%%%%%%%%%%%%%%%%%%%%%%%%%%%%%%%%%%%%%%%%%%%%%%%%

\hhline{|-|-|-~}
	 \multicolumn{1}{|p{\gnumericColA}|}%
	{\gnumericPB{\centering}\gnumbox{$f(s_1)$}}
	&\multicolumn{1}{p{\gnumericColB}|}%
	{\gnumericPB{\centering}\gnumbox{$f(s_2)$}}
	&\multicolumn{1}{p{\gnumericColC}|}%
	{\gnumericPB{\centering}\gnumbox{$f(s_1)+f(s_2$)}}
	&
\\
\hhline{|---|~}
	 \multicolumn{1}{|p{\gnumericColA}|}%
	{\gnumericPB{\centering}\gnumbox{1}}
	&\multicolumn{1}{p{\gnumericColB}|}%
	{\gnumericPB{\centering}\gnumbox{1}}
	&\multicolumn{1}{p{\gnumericColC}|}%
	{\gnumericPB{\centering}\gnumbox{1}}
	&
\\
\hhline{|---|~}
	 \multicolumn{1}{|p{\gnumericColA}|}%
	{\gnumericPB{\centering}\gnumbox{1}}
	&\multicolumn{1}{p{\gnumericColB}|}%
	{\gnumericPB{\centering}\gnumbox{-1}}
	&\multicolumn{1}{p{\gnumericColC}|}%
	{\gnumericPB{\centering}\gnumbox{-1}}
	&
\\
\hhline{|---|~}
	 \multicolumn{1}{|p{\gnumericColA}|}%
	{\gnumericPB{\centering}\gnumbox{-1}}
	&\multicolumn{1}{p{\gnumericColB}|}%
	{\gnumericPB{\centering}\gnumbox{-1}}
	&\multicolumn{1}{p{\gnumericColC}|}%
	{\gnumericPB{\centering}\gnumbox{1}}
	&
\\
\hhline{|---|~}
	 \multicolumn{1}{|p{\gnumericColA}|}%
	{\gnumericPB{\centering}\gnumbox{-1}}
	&\multicolumn{1}{p{\gnumericColB}|}%
	{\gnumericPB{\centering}\gnumbox{1}}
	&\multicolumn{1}{p{\gnumericColC}|}%
	{\gnumericPB{\centering}\gnumbox{-1}}
	&
\\
\hhline{|-|-|-|~}
\end{tabular}

\ifthenelse{\isundefined{\languageshorthands}}{}{\languageshorthands{\languagename}}
\gnumericTableEnd

			\caption{}
			\label{table:1.3.8}
		\end{table}
	\item No. If $s_1,s_2$ are odd, 
	\begin{align}
		s_1s_2 & \text{ odd}
		\\
		f(s_1s_2) &= -1 \ne f(s_1)f(s_2)
	\end{align}
\end{enumerate}
\item Let $S$ be the set of all real numbers.  Define 
	\begin{align}
		f&:S \rightarrow S | f(s) = s^2,
		\\
		g&:S \rightarrow S | g(s) = s+1,
	\end{align}
\begin{enumerate}
	\item Find $f \circ g$.
	\item Find $g \circ f$.
	\item Is $f \circ g=g \circ f$?
\end{enumerate}
\solution 
\begin{enumerate}
	\item 
	\begin{align}
		\label{eq:1.3.9-1}
		\brak{f \circ g}(s) = \brak{s+1}^2
	\end{align}
	\item 
	\begin{align}
		\label{eq:1.3.9-2}
		\brak{g \circ f}(s) = s^2+1
	\end{align}
	\item From 
		\eqref{eq:1.3.9-2}
		and
		\eqref{eq:1.3.9-2}
$f \circ g \ne g \circ f$.
\end{enumerate}
\item Let $S$ be the set of all real numbers and for $a, b \in S$, where $a \ne 0$; define $f_{a,b}(s) = as + b$.
\begin{enumerate}
	\item  Show that $f_{a,b} \circ f_{c,d} = f_{u,v}$ for some real $u, v$. Give explicit values for $u, v$ in terms of $a, b, c$ and $d$.
	\item  Is $f_{a,b} \circ f_{c,d} = f_{c,d} \circ f_{a,b}$ always?
	\item  Find all $f_{a,b}$ such that $f_{a,b} \circ f_{1,1} = f_{1,1} \circ f_{a,b}$.
	\item  Show that $f^{-1}$ exists and find its form.
\end{enumerate}
\solution 
\begin{enumerate}
	\item  
		\begin{align}
			\label{eq:1.3.10-1}
			f_{a,b} \circ f_{c,d} &= c\brak{as+b}+d
			\\
			&=cas+cb+d 
			\\
			&= us+v
			\\
			\implies u = ca, v = bc + d
		\end{align}
	\item  From
			\eqref{eq:1.3.10-1},
		\begin{align}
			\label{eq:1.3.10-2}
			f_{c,d}\circ f_{a,b} 
			&=cas+ad+ b
		\end{align}
		Thus, from 
			\eqref{eq:1.3.10-1}
			and 
			\eqref{eq:1.3.10-2}
		\begin{align}
			f_{a,b} \circ f_{c,d} &= f_{c,d} \circ f_{a,b} 
			\\
			\implies bc+d = ad+b
			\label{eq:1.3.10-3}
		\end{align}
	\item  From 
			\eqref{eq:1.3.10-3},
		\begin{align}
			f_{a,b} \circ f_{1,1} &= f_{1,1} \circ f_{a,b}
			\\
			\implies as+b+1 &= as+a+b
			\\
			\text{or, } a = 1.
		\end{align}
		Thus, 
		\begin{align}
			f_{a,b}  = s+b
		\end{align}
	\item  From the definition,
		\begin{align}
			f_{a,b}(s)  &= as+b
			\\
			\implies s &=\frac{f_{a,b}(s) -b}{a}
			\\
			\text{or, }f^{-1}_{a,b}(s) &= \frac{s-b}{a}
		\end{align}
\end{enumerate}
\item Let $S$ be the set of all positive integers. Define $f: S  \rightarrow S$ by $f(l) = 2, f(2) = 3, f(3) = 1$ and $f(s) = s$ for any other $s \in S$. Show that $f o f o f = i_S$. What is $f^{-1}$ in this case?
	\\
	\solution For
$s \in \cbrak{1,2,3}$, it is obvious.  For
	$s \notin \cbrak{1,2,3}$,
		\begin{align}
			\brak{f\circ f} (s) &= f(s) = s
			\\
			\implies \brak{f\circ f\circ f} (s) &=  s \qed
		\end{align}

\end{enumerate}
