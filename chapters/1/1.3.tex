\renewcommand{\theequation}{\theenumi}

\begin{enumerate}[label=\arabic*.,ref=\thesubsection.\theenumi]
		\numberwithin{equation}{enumi}

	\item For the given sets $S, T$ determine if a mapping $f:S\to T$ is clearly and unambiguously defined; if not, say why not.
		\label{prob:1.3.1}
\begin{enumerate}
	\item $S =$ set of all women, $T = $ set of all men, $f(s) = $ husband of s.
	\item $S = $ set of all positive integers, $T = S, f(s) = s-1$.
	\item $S$ = set of positive integers, $T$ = set of nonnegative integers, $f(s) = s -1$.
	\item $S$ = set of nonnegative integers, $T = S, f(s) = s - 1$.
	\item $S$ = set of all integers, $T = S, f(s) = s - 1$.
	\item  $S$ =set of all real numbers, $T = S, f(s) = \sqrt{s}$.
	\item	$S$ = set of all positive real numbers, $T = S, f(s) = \sqrt{s}$.	
\end{enumerate}
\solution
\begin{enumerate}
	\item Not all women have husbands.  So the mapping is not clearly defined.
	\item For every integer $s, s-1$ is an integer.  So the mapping is defined.
	\item $0 \notin S$, so the mapping is defined.
	\item $f(0) = -1 \notin S$.  So the mapping is not defined.
	\item $f(-1) \notin S$, so the mapping is not defined.
	\item $f(s) \in S \forall S$.  So the mapping is defined.
\end{enumerate}
\item 
	In those parts of Problem  
		\ref{prob:1.3.1}
where $f$ does define a function, determine if it is 1-1, onto, or both.
\solution
\begin{enumerate}
	\item For $f(s) = s-1, s \in \mathbb{Z}$, the mapping is a bijection.
	\item For $s \in \mathbb{N}, f(s) = s-1 \in  \mathbb{W}$, the mapping is a bijection.
	\item For $s \in S, f(s) \in S$ and vice-versa.  So the mapping is a bijection.
\end{enumerate}
\item If $f$ is a 1-1 mapping of $S$ onto $T$, prove that $f^{- 1}$ is a 1-1 mapping of $T$ onto $S$.
	\\
		\solution  By definition,  
		\begin{align}
			\label{eq:1.3.one-one}
			\begin{split}
	s_1 = s_2 \in S	\implies	f(s_1) = f(s_2) \in T
	\\
			t_1 = t_2 \in T \implies \exists s_1 = s_2 \in S \ni f(s_1) = f(s_2).
			\end{split}
		\end{align}
Let $g = f^{-1}$. Then, 
		\begin{align}
			f(s_i) = t_i \implies g(t_i) = s_i.
		\end{align}
		From 
			\eqref{eq:1.3.one-one}, 
		\begin{align}
			\label{eq:1.3.one-one-inv}
			\begin{split}
				g(t_1) = g(t_2) \in S	\implies	 t_1= t_2 \in T
	\\
			t_1 = t_2 \in T \implies \exists g(t_1) = g(t_2) \in S 
			\end{split}
		\end{align}
			\eqref{eq:1.3.one-one-inv} shows that $g = f^{-1}$ is also 1-1.
		
\end{enumerate}
