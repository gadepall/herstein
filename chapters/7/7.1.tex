\renewcommand{\theequation}{\theenumi}
\begin{enumerate}[label=\arabic*.,ref=\thesubsection.\theenumi]
\numberwithin{equation}{enumi}
\item Show that the eccentricity of a rectangular hyperbola is $\sqrt{2}$
\item Prove that the locus of the centre of a circle which touches two
given non-concentric circles of unequal radii is an ellipse or a hyperbola.
\item $ABC$ is a triangle in which $C$ is a right angle.  On $CA$, $CB$ points $P, Q$ are
taken such that $CP.CQ=CA.CB$.  Prove that the locus of the centroid of the trianlge $CPQ$ is a rectangular hyperbola.
\item Show that the tangents to a rectangular hyperbola at the extremities of its latera
recta pass through the vertices of the conjugate hyperbola.
\item $P$ is a point on a rectangular hyperbola whose centre is $C$ and a 
line is drawn through $C$ perpendicular to $CP$.  Through
$Q$, any point on the curve, lines are drawn parallel to the asymptotes meeting
this line in $L$, $M$, show that $LPM$ is a right angle.
\item Prove that the intercept on any tangent to a hyperbola, made by its asymptotes, subtends a constant
angle at either focus.
\item Prove that the line $lx+my=n$ touches the rectangular hyperbola $xy=c^2$, if
\begin{align*}
n^2 = 4lmc^2
\end{align*}
\item A tangent to a hyperbola of foci $S$, $S_1$ meets the asymptotes
in $L$, $L_1$.  Prove that the points $S, L, S_1, L_1$ are concyclic.
\item The tangents at two points $P$, $P_1$ on a rectangular hyperbola meet an asymptote in $L, L_1$ and
$P,P_1$ meets it in $K$.  Prove that
\begin{align*}
LK = KL_1
\end{align*}
\item Prove that conjugate diameters of a rectangular hyperbola are equally inclined to the asymptotes.
\item Prove that the polars of a point with regard to two conjugate hyperbolas are parallel and equidistant from the
centre.
\item Prove that in a rectangular hyperbola any chord $PP_1$ subtends at the ends of a diameter $AA_1$ 
angles which are either equal or suplementary.
\item Prove that if $CP$, $CD$ are conjugate radii of a hyperbola the orthocentre of
the triangle $CPD$ lies on the line $ax=by$.
\item Prove that, if the tangent at $P$ to a rectanglular hyperbola meets the asymptotes in $L$, $L_1$ and
the normal at $P$ meets the transverse axis in $G$, then $LGL_1$ is a right angle.
\item Prove that if an ellipse and a hyperbola have the same foci they cut one another at right
angles.
\item The perpendiculars from a point $P$ to the axes meet them in $M, N$, and the perpendicular bisector
of $MN$ passes through a fixed point $C$ on one of the axes.  Prove that the locus of $P$ is a rectangular
hyperbola with centre at $C$.
\item Show that the locus of the middle point of a chord of the rectangular 
hyperbola $xy=c^2$ of constant length $2a$ is
\begin{align*}
\brak{xy-c^2}\brak{x^2+y^2} = a^2xy
\end{align*}
\item Prove that, if the position of a point on  a rectangular hyperbola is determined by the
variable $\theta$ where $x = c \tan\theta$, $y = c\cot\theta$, the locus of the intersection of tangents at the 
points $\theta$, $\theta+\alpha$ is
\begin{align*}
4\brak{c^2-xy} = \brak{x+y}^2\tan^2\alpha
\end{align*}
$\alpha$ being a constant angle.
\item Tangents at right angles are drawn to a rectangular hyperbola and its
conjugate.  Show that they cut either asymptote in two points
$K$, $K_1$ such that the rectangle $CK.CK_1$ is equal to twice the square on
the semi-axis, where $C$ is the centre.
\item Prove that the line joining the feet of the perpendiculars drawn to a
pair of conjugate diameters of a rectangular hyperbola from any point $P$ on the
hyperbola is parallel to the normal at $P$.
\item  The normal at $P_1$ on the hyperbola $xy-c^2=0$ meets the curve again at $P_2$, the normal at $P_2$
 meets the curve again at $P_3$
and so on.  Prove that if $y_1,y_2,\dots, y_{n+1}$ are the ordinates of these points respectively,
\begin{align*}
y_1^2y_2 = y_2^2y_3 = \dots = y_n^2y_{n+1} = c^4.
\end{align*}
\item If $PN$ be the ordinate and $PG$ the normal of a point $P$ of a hyperbola, whose centre is $C$, and the tangent
at $P$ intersect the asymptotes at $L$ and $L_1$, show that half the sum of $CL$ and $CL_1$ is the mean
proportional between $CN$ and $CG$.
\item At the point of intersection of the rectangular hyperbola $xy=k^2$, and of the parabola $y^2=4ax$, the tangents
to the hyperbola and parabola make angles $\theta$ and $\phi$ respectively with the axis of $x$.  Prove that
\begin{align*}
\tan\theta = -2\tan\phi.
\end{align*}
\item Prove that, if $A, B, C$ are three points on a rectangular hyperbola, the curve passes through the orthocentre
of the triangle $ABC$.
\item Prove that, in a rectangular hyperbola, the product of the focal distances of a point is equal
to the square of the distance of the point from the centre.
\item Prove that, if $\brak{c\tan\theta,c\cot\theta}$ and $\brak{c\tan\theta_1,c\cot\theta_1}$ are two points on 
the hyperbola $xy=c^2$, and $\theta+\theta_1$ is constant, then the chord
joining the points passes through a fixed point on the conjugate axis of the hyperbola.
\item Show that the point whose coordinates are
\begin{align*}
\frac{a}{2}\brak{t+\frac{1}{t}}, \frac{b}{2}\brak{t-\frac{1}{t}}
\end{align*}
lies on the hyperbola $\frac{x^2}{a}-\frac{y^2}{b^2} = 1$. Prove that, if $C$ is the centre of the hyperbola and $S$ is either focus 
and if the tangent at the above point meets the asymptote $\frac{x}{a}=\frac{y}{b}$ at $X$ and meets the asymptote $\frac{x}{a}=-\frac{y}{b}$
 at $Y$, then
\begin{align*}
t = \frac{CX}{CS} = \frac{CS}{CY}
\end{align*}
\item From any point on the normal at a given point $A$ on a rectangular hyperbola the other three normals to
the curve are drawn.  Sow that the centroid of the feet of these three normals lies on the diameter of the hyperbola
parallel to the normal at $A$.
\item A circle is drawn passing through any point $P$ on the hyperbola $\frac{x^2}{a^2}-\frac{y^2}{b^2}=1$ and through the ends $A$, $A_1$ of
the transverse axis.  The ordinate $NP$ is produced to  meet the circle again in $Q$.  Prove that, as
the position of $P$ on the hyperbola varies, the locus of $Q$ is the hyperbola $\frac{x^2}{a^2}-\frac{y^2}{b_1^2} = 1$, where $b_1 = \frac{a^2}{b}$.
\end{enumerate}
